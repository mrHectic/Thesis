\section*{Summary}

Currently, there is no system in place to autonomously evaluate student code used in E-Design and Computer System modules. The code is often widely varying, stored in nested repositories and laborious to assess manually. This project aims to alleviate the problem by introducing automation in the form of emulation and code evaluation. 
\\\\
This is done by firstly, investigating possible emulation solutions. High fidelity emulators serve as hardware-independent platforms on which code can be run. By mimicking simple MCU functionality within an emulator, the doors to possible future automation using emulation are left ajar.
\\\\
Once an emulation solution has been achieved, a system whereby student code can autonomously be investigated is warranted. This project will outline the design and implementation of such a system. 

\section*{Opsomming}
Huidiglik is daar geen stelsel in plek om studentekode wat in E-Ontwerp en Rekenaarstelselmodules gebruik word onafhanklik te evalueer nie. Die kode is meestal wisselend, word in beneste bewaarplekke gestoor en is moeisaam om handmatig te assesseer. Hierdie projek se doel is om die bogenoemde probleem te verlig deur outomatisering in te stel in die vorm van emulasie en kode-evaluaring. 
\\\\
Dit word gedoen deur eerstens moontlike emuleringsoplossings te ondersoek. waarHoë-getrouheid emulators dien as platforms wat onafhanklik van hardeware is waarop kode kan loop. Deur die eenvoudige MCU funksionalitiet in ‘n emulator na te boots, kan die moontlikheid van outomatisering wat emulasie benut in die toekoms ondersoek word. 
\\\\
Nadat 'n emulasie oplossing bereik word, kan ‘n stelsel te staan kom waarvolgens studentekode outonomies ondersoek kan word. Hierdie projek sal die ontwerp en implementasie van so ‘n stelsel uiteensit.
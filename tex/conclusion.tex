

\chapter*{6 Conclusion}
\label{6conc}
\addcontentsline{toc}{chapter}{6 Conclusion}
\setcounter{chapter}{6}
\setcounter{section}{0}
\setcounter{figure}{0}
\setcounter{table}{0}

This chapter highlights and reiterates the achieved objectives, results and future recommendations of the project. It should be read in conjunction with \textbf{\nameref{intro}} and is reconcilable with the mentioned chapter.

\section{Objectives achieved}
The problem statement as illustrated in \textbf{\ref{ps} \nameref{ps}}, is achieved by fulfilling the objectives of this project. If these objectives are indeed adequately fulfilled, the problem stated will have been resolved.

\subsection{Emulator/simulator investigation}
It was illustrated in \textbf{\nameref{2emul}} and \textbf{\ref{emuRes} \nameref{emuRes}}, that it is indeed possible, with enough effort, to mimic real-world MCU behaviour on a chosen emulator. The implications of this are readily apparent.
\\\\
If simple functionality (such as a blinking LED) can be replicated within the QEMU emulator as detailed in \textbf{\nameref{2emul}}, any functionality could theoretically be replicated to some degree of accuracy. 
\\\\
It was clearly illustrated in \textbf{\nameref{2emul}} that the blinking LED functionality can be replicated in an emulated environment. The objective as stated in \textbf{\ref{emInvestObj} \nameref{emInvestObj}} has thus been achieved.

\subsection{Autonomous high-level code structure evaluation}
\textbf{\ref{highLevObj} \nameref{highLevObj}} states a second objective of this project. Once this second objective has been achieved, the problem as stated in \textbf{\ref{ps} \nameref{ps}} has been fully resolved.
\\\\
The subject of \textbf{\nameref{system3}} and \textbf{\nameref{4detailedd}}, is the achievement of this objective. It is quite apparent from the previously mentioned chapters, in conjunction with \textbf{\ref{codRes} \nameref{codRes}}, that this objective has been achieved. 
\\\\
Varying student code can easily be evaluated in terms of MCU configurations in an autonomous way. It is important to note that pin and clock configurations were chosen to illustrate this achieved aim. It is however possible to extend this functionality to any high level code structures by simply tweaking the python code (specifically the regular expressions discussed) to search for different patterns.
\\\\
The second objective of this project has thus been met. The problem statement has been resolved and a novel solution has been applied to a dynamic problem.

\section{Recommendations}

\color{green} I am not sure what to put into this section Dr Barnard ro we\color{black}



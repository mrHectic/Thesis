

\chapter*{6 Conclusion}
\label{6conc}
\addcontentsline{toc}{chapter}{6 Conclusion}
\setcounter{chapter}{6}
\setcounter{section}{0}
\setcounter{figure}{0}
\setcounter{table}{0}

This chapter highlights and reiterates the achieved objectives, results and future recommendations of the project. It should be read in conjunction with \textbf{\nameref{intro}} and is reconcilable with the mentioned chapter.

\section{Objectives achieved}
The problem statement as illustrated in \textbf{\ref{ps} \nameref{ps}}, is achieved by fulfilling the objectives of this project. If these objectives are indeed adequately fulfilled, the problem stated will have been resolved.

\subsection{Emulator/simulator investigation}
It was illustrated in \textbf{\nameref{2emul}} and \textbf{\ref{emuRes} \nameref{emuRes}}, that it is indeed possible, with enough effort, to mimic real-world MCU behaviour on a chosen emulator. The implications of this are readily apparent.
\\\\
If simple functionality (such as a blinking LED) can be replicated within the QEMU emulator as detailed in \textbf{\nameref{2emul}}, any functionality could theoretically be replicated to some degree of accuracy. 
\\\\
It was clearly illustrated in \textbf{\nameref{2emul}} that the blinking LED functionality can be replicated in an emulated environment. The objective as stated in \textbf{\ref{emInvestObj} \nameref{emInvestObj}} has thus been achieved.

\subsection{Autonomous high-level code structure evaluation}
\textbf{\ref{highLevObj} \nameref{highLevObj}} states a second objective of this project. Once this second objective has been achieved, the problem as stated in \textbf{\ref{ps} \nameref{ps}} has been fully resolved.
\\\\
The subject of \textbf{\nameref{system3}} and \textbf{\nameref{4detailedd}}, is the achievement of this objective. It is quite apparent from the previously mentioned chapters, in conjunction with \textbf{\ref{codRes} \nameref{codRes}}, that this objective has been achieved. 
\\\\
Varying student code can easily be evaluated in terms of MCU configurations in an autonomous way. It is important to note that pin and clock configurations were chosen to illustrate this achieved aim. It is however possible to extend this functionality to any high level code structures by simply tweaking the python code (specifically the regular expressions discussed) to search for different patterns.
\\\\
The second objective of this project has thus been met. The problem statement has been resolved and a novel solution has been applied to a dynamic problem.

\section{Recommendations}

As described in \textbf{\ref{scOfWrk} \nameref{scOfWrk}}, and as a result of the limited time available for this project, certain constraints had to be introduced. For future work, in addressing the topic at hand, it is recommended that some of these limitations are lifted. 
\\\\
Firstly, it is of note that the list of supported MCUs for various emulators will change over time. It is, therefore, recommended that future projects in a similar vain to this one, re-assess the list of supported MCUs for the various emulators. Of particular interest is the planned future support of the STM32 Nucleo range by the xPack Arm QEMU Project. In conjunction with the aforementioned, it is recommended that the other emulators mentioned in this project, be further investigated as time passes and development continues.
\\\\
In addition, the chosen functionality to recreate in the emulator involved a flashing LED. It was indeed outside the scope of this project to recreate more advanced functionalities within the described frameworks of \textbf{\nameref{2emul}}. More functionality can thus be introduced by future endeavours (given that solutions supporting more peripherals are discovered).
\\\\
It can be seen from \textbf{\ref{codRes} \nameref{codRes}}, that a simple terminal is used to guide the user in the process of autonomous code evaluation. This was done to facilitate the process of automation, but required some manual interaction by the user. It is hence recommended, for the sake of automation, that future projects reduce the extent of these manual interactions. This can be done using a GUI or an autonomous execution chain.
\\\\
Furthermore, the evaluation of student code is limited in this project. Evaluation occurred for pin and clock configurations exclusively due to time and scope constraints. It is recommended that succeeding investigations extend the scope of the evaluation to include further configurations (like UART). 
\\\\
Lastly, this project addressed the problem statement in a two-part, mutually exclusive way. The investigation of the emulator was not dependant on, nor relevant to, the evaluated code. A future project might, therefore, consider recreating student MCU functionality on a chosen emulator in its totality. Implementing the aforementioned functionality is, indeed, a very extensive task and should perhaps be the topic of an entire project.



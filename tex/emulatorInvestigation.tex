

\chapter*{2 Emulator Investigation}
\addcontentsline{toc}{chapter}{2 Emulator Investigation}
\setcounter{chapter}{2}
\setcounter{section}{0}
\setcounter{figure}{0}
\setcounter{table}{0}
\label{2emul}

\section{Emulators and Simulators}
\label{emuVsSim}
To address the problem adequately, one instruction set must be emulated on a platform compatible only with another instruction set. In this case, C code written for ARM architectures must be emulated on a \hyperref[listAbr]{PC} using CISC instruction sets.
\\\\
To achieve this, an ARM machine must be virtually created on the host machine. This can be done either through emulation or simulation. Emulation aims to mimic the target architecture (ARM) as closely as possible on the host platform (x86 Intel/AMD) at the cost of simplicity. It also provides a much less general, more platform dependent solution. In contrast, simulation aims to duplicate only high-level behaviour and is not very specific. It is not suitable for high fidelity application \cite{Chris}. The table below illustrates some key comparisons.

\begin{table}[H]
\begin{tabular}{ |c|c|c| } 
 \hline
 Goals & Emulation suitability & Simulation suitability \\ 
 \hline
 High Fidelity & Full & Partial \\ 
 Close to Real-time operation& Partial & None \\ 
 For studying principles & Partial & Full\\ 
 For recreating behavior & Full & Partial\\ 
 For general solutions & None & Full\\ 
 \hline
\end{tabular}
\caption{Emulator vs Simulators}
\label{table:1}
\end{table}

The above table illustrates the key differences in emulation and simulation and the suitability of each. As can be seen, high fidelity and operation close to real time. Only emulators are suitable. Furthermore, because behaviour is indeed being recreated and the scope has been narrowed as per \textbf{1.4 \nameref{scOfWrk}}, a simulator is not warranted. 
\\\\
It therefore becomes the goal of this investigation to find a suitable emulator for the task at hand. In the following sections emulators will be investigated. It will become apparent that these solutions are extremely specific in most cases and target a small range of \hyperref[listAbr]{MCU}s only. The choice of emulator is therefore extremely important and is precursor to automating any code evaluation.

\section{GXemul}
\label{GXemul}
The first emulator under investigation is GXemul. It is promising in that some machines using ARM processors have been emulated within its framework. Furthermore, the project is completely open-source. \cite{Gavare}
\\\\
GXemul is suitable for low-level programming courses as well as operating system courses and aims to be a learning tool more than anything else. Its framework revolves around specific machines mentioned in the pertaining documentation \cite{Gavare} \cite{gavareEmail}. Since the specific \hyperref[listAbr]{ARM} boards used in the engineering modules of importance are not supported (No Arduino or STM32 MCU support), the source-code would need to be altered to support the MCU. It is far beyond the scope of this project to code an emulators basic functioning, but it is indeed interesting that it can, theoretically be done in GXemul.
\\\\
Email correspondence with Anders Gavare (the creator of GXemul) confirmed many of the aforementioned observation and additionally provided insight into a further limitation of GXemul: It prefers host operating systems Linux and FreeBSD. Although emulation on other operating systems are indeed possible, they are not explicitly supported \cite{gavareEmail}. This makes GXemul non-ideal for the purposes of this project. It is ultimately too specific, educationally focused and lacks support for the most common platforms.
 
\section{OVPsim}
\label{OVPsim}
OVPsim provides emulation of various target architectures on multiple host platforms and includes support for multiprocessors. The OVPsim project makes use of public \hyperref[listAbr]{API}s allowing the creation of custom processors and hardware arrangements \cite{Imperas2020}. It is indeed a very versatile and complete solution.
\\\\
In contrast to GXemul, OVPsim is widely used on the most popular \hyperref[listAbr]{OS}s. This makes it suitable for the problem at hand, due to the scope being limited to Windows \hyperref[listAbr]{PC}s (with support for Linux, taking an important caveat into account)(see \textbf{1.4 \nameref{scOfWrk}}).
\\\\
Even though OVPsim seems to be suitable, very little documentation can be found regarding the process of implementing a target architecture on it. The team behind the project are not available for communication and internet forums are largely devoid of any information regarding this particular emulator. 
\\\\
A disconnect thus exists between the suitability of this emulator and the usability of it. A solution might be possible using OVPsim but there is no clear path to that solution. Further investigation into alternative emulator solutions are thus warranted.

\section{QEMU}
\label{qemu}
QEMU is a full-system emulator. It is an ubiquitous solution when it comes to \hyperref[listAbr]{ARM} processors. It is constantly being developed by the team behind it and documentation is readily available \cite{QEMU}.
\\\\
Furthermore, QEMU supports two specific \hyperref[listAbr]{ARM} Cortex-M4  \hyperref[listAbr]{MCU}s, namely the Stellaris LM3S811EVB and the Stellaris LM3S6965EVB. This makes it a very promising option, as two specific Cortex-M4 \hyperref[listAbr]{MCU}s are fully supported in terms of functionality \cite{QEMU}\cite{QEMUarm}. The previously investigated emulators, in contrast, had very little support in terms of full \hyperref[listAbr]{MCU} recreation and offered only partial solutions.
\\\\
QEMU is furthermore, a cross-platform project and therefore supports (and will continue to support for the foreseeable future) Windows \hyperref[listAbr]{PC}s. This makes it viable with the narrowing of the project scope occurring in \textbf{1.4 \nameref{scOfWrk}}. 
\\\\
It is important to note that the two ARM Cortex-M series \hyperref[listAbr]{MCU}s supported by the QEMU project are not boards usually used in the pertaining engineering modules mentioned earlier. These two boards are not aimed at low-power applications and are suited more for industrial applications than educational ones \cite{TexasInstruments2014}. Furthermore, the two boards supported by the QEMU emulator are outdated and have been superseded by the manufacturer. It becomes apparent that the QEMU emulator, in its current state, is not ideal for the purposes of this project. Indeed an emulator that supports STM32s (the \hyperref[listAbr]{MCU}s preferred for use in the pertaining engineering modules) would be preferable. Due to the fact that QEMU is open-source, it can be investigated whether these STM32 \hyperref[listAbr]{MCU}s have been implemented in projects built within the QEMU framework.

\section{The xPack Project}
\label{xpack}
The xPack Project is a modification of the the QEMU project. It modifies the open-source QEMU emulator in such a way that a wider array of Cortex-M cores are supported. This allows the emulation of various \hyperref[listAbr]{MCU}s, most notably some \hyperref[listAbr]{MCU}s in the STM32 family \cite{xPack}.
\\\\
The support, albeit partial, of the STM32 family is cardinal as ST-link`s STM32  range is the \hyperref[listAbr]{MCU} range mostly used in the pertaining engineering modules. If it can be shown that simple functionality, programmed for a real-world \hyperref[listAbr]{MCU}, can be replicated in an emulated environment, part of the problem statement as per \textbf{1.1 \nameref{ps}}, would be addressed. The real-world \hyperref[listAbr]{MCU}s available (for the aforementioned programming) are the STM32F411E and STM32F334R8-Nucleo as they were prescribed in my engineering modules and are thus in my possession.
\\\\
Since STM32 \hyperref[listAbr]{MCU}s are widely used in computer systems and design modules, an emulator that supports these boards specifically will be preferred. 
\\\\








%Whilst emulation mimics a pertaining architecture relatively closely, simulation does so more loosely. Emulation attempts to duplicate one device as accurately as possible in another environment. Simulation, by contrast, is not concerned with low-level duplication of devices, but instead mimics high-level behaviour.\cite{Chris}
\part{Emulator Investigation}
\label{part1}
%%\addcontentsline{toc}{chapter}{2 Emulator Investigation}
In order to solve the problem as stated in the \color{blue}introduction\color{black}, a solution that allows \hyperref[listAbr]{ARM}-type processors to be mimicked on standard \hyperref[listAbr]{PC} processors is needed. Since most \hyperref[listAbr]{PC}s make use of either intel\textsuperscript{{\tiny{\textregistered}}} or AMD\textsuperscript{{\tiny{\textregistered}}} \hyperref[listAbr]{CPU}'s (which are x86 based architectures - an ubiquitous iteration of \hyperref[listAbr]{CISC}), emulation or simulation is indeed required. This is because, on a machine level, \hyperref[listAbr]{CISC} architecture is incompatible with \hyperref[listAbr]{RISC} assembly language.
\\\\
It has been established that \hyperref[listAbr]{RISC} architecture will be mimicked on \hyperref[listAbr]{CISC} machines in order to evaluate student code. The degree of mimicry needed depends largely on the application. Whilst emulation mimics a pertaining architecture relatively closely, simulation does so more loosely. Emulation (and subsequently emulators) attempts to duplicate one device as accurately as possible in another environment. Simulation, by contrast, is not concerned with low-level duplication of devices, but instead mimics high-level behaviour. \cite{Chris}
\\\\
For the evaluation of student specific code, a simulator will suffice as low-level architecture need not be 
